\chapter{Theory: Method for Measuring Correlation in the Velocity Covariance Matrix (title is a work in progress)}

In this section I will be writing about how we will be measuring correlation of peculiar velocities using a new method we are developing in this thesis. We begin with a given a set of distance and redshift measurements, from which we can calculate the peculiar velocity of each object as detailed in the background. We can then organize these measurements to represent a velocity field $v(r)$ where we can use the galaxies to trace the large-scale motion. Performing the Fourier transform of the peculiar velocity field allows us to measure the velocity power spectrum of the field. Since the velocity power spectrum is directly related to the density power spectrum, we can observe large-scale perturbations in the density spectrum and identify their origin. Since the density power spectrum relies on cosmological parameters, to estimate the bulk flow we must perform a likelihood analysis on the value of the cosmological parameters $\Omega_m$, $h$ and $\Omega_b$ based on on our peculiar velocity field data. The likelihood function involves a velocity covariance matrix for each peculiar velocity in the field. The method used in the past weighted the contribution of peculiar velocities to the bulk flow based on which scales they probe as measured in the velocity power spectrum. This method was useful in reducing the influence of small scale motions, but the bulk flow they measured disagreed with the value estimated from the standard cosmological model ($\Lambda$CDM model) using the cosmic microwave background (CMB) data. Thus, we will be reassessing the inclusion of the small-scale influences by using the velocity covariance matrix to probe the best cosmological parameters that match the data for measuring the bulk flow.