\chapter{Introduction}

In localized regions of space, smaller scale influences such as gravitational attraction dominate pulling matter towards regions of higher density. A simple example of this would be a galaxy  being pulled towards a larger, more massive galaxy. While this behavior is most certainly still present at scales much larger than individual galaxies, the ``motion" that dominates in this realm is due cosmological expansion which is often referred to more simply as the ``stretching" of space between objects. While the driving factor for this large scale motion is still in principle due to gravity, what we see is that objects in space are being stretched away from each other often faster than they are moving toward regions of higher density. However, the motion towards regions of higher density at large scales can be seen as deviations from an idealized isotropic model of cosmological expansion.  These aberrations in the model that indicate large scale flows in contrast to the linear growth of the universe can tell us more about the origin of the universe from the Big Bang and what the behavior of the universe may look like in the future. In this thesis we concern ourselves with the influence of the deviations from the model of cosmological expansion and propose a method to adequately model the large scale motions of the universe.
